\documentclass[
	12pt,				% tamanho da fonte
	openright,			% capítulos começam em pág ímpar (insere página vazia caso preciso)
	oneside,			% para impressão apenas em um lado do papel
	a4paper,			% tamanho do papel.
	brazil				% o último idioma é o principal do documento
	]{abntex2}

% ---
% Pacotes básicos
% ---

\usepackage{lmodern}			% Usa a fonte Latin Modern
\usepackage[T1]{fontenc}		% Selecao de codigos de fonte.
\usepackage[utf8]{inputenc}		% Codificacao do documento (conversão automática dos acentos)
\usepackage{lastpage}			% Usado pela Ficha catalográfica
\usepackage{indentfirst}		% Indenta o primeiro parágrafo de cada seção.
\setlength{\parindent}{1.5cm}   % Espaçamento de 1,5cm do parágrafo
\usepackage{color}				% Controle das cores
\usepackage{graphicx}			% Inclusão de gráficos
\usepackage{microtype} 			% para melhorias de justificação
\usepackage{lipsum}				% para geração de dummy text
\usepackage[alf]{abntex2cite}	% Citações padrão ABNT
\usepackage[table,xcdraw]{xcolor}% Cédula colorida em tabelas
\usepackage{pdflscape}          % Rotaciona página
% \usepackage{Capa}               % Capa e folha de rosto com as modificações da Suely

\usepackage{amsmath, amssymb, amsthm}

% -----------------------------------------------------------------------
\renewcommand{\imprimircapa}{%
  \begin{capa}%
    \center
    \ABNTEXchapterfont\Large \imprimirinstituicao \\
    \vspace*{1cm}
    {\ABNTEXchapterfont\large\imprimirautor}
    \vfill
    \ABNTEXchapterfont\bfseries\LARGE\imprimirtitulo
    \vfill
    \large\imprimirlocal
    \par
    \large\imprimirdata
    \vspace*{1cm}
  \end{capa}
}


\makeatletter
\renewcommand{\folhaderostocontent}{
  \begin{center}
    {\ABNTEXchapterfont\large\imprimirautor}
    \vspace*{\fill}\vspace*{\fill}
    \begin{center}
      \ABNTEXchapterfont\bfseries\Large\imprimirtitulo
    \end{center}
    \vspace*{\fill}
    \abntex@ifnotempty{\imprimirpreambulo}{%
      \hspace{.45\textwidth}
      \begin{minipage}{.5\textwidth}
        \SingleSpacing
        \imprimirpreambulo \hfill
      \end{minipage}%
      \vspace{0.5cm}

      \hspace{.45\textwidth}
      \begin{minipage}{.5\textwidth}
        \imprimirorientadorRotulo~\imprimirorientador \hfill
      \end{minipage}%
      \vspace*{\fill}
    }%
    \vfill
    {\large\imprimirlocal}
    \par
    {\large\imprimirdata}
    \vspace*{1cm}
  \end{center}
}
\makeatother
%-----------------------------------------------------------------------

\usepackage{color}
\usepackage{fancyvrb}
\newcommand{\VerbBar}{|}
\newcommand{\VERB}{\Verb[commandchars=\\\{\}]}
\DefineVerbatimEnvironment{Highlighting}{Verbatim}{commandchars=\\\{\}}
% Add ',fontsize=\small' for more characters per line
\usepackage{framed}
\definecolor{shadecolor}{RGB}{248,248,248}
\newenvironment{Shaded}{\begin{snugshade}}{\end{snugshade}}
\newcommand{\AlertTok}[1]{\textcolor[rgb]{0.94,0.16,0.16}{#1}}
\newcommand{\AnnotationTok}[1]{\textcolor[rgb]{0.56,0.35,0.01}{\textbf{\textit{#1}}}}
\newcommand{\AttributeTok}[1]{\textcolor[rgb]{0.13,0.29,0.53}{#1}}
\newcommand{\BaseNTok}[1]{\textcolor[rgb]{0.00,0.00,0.81}{#1}}
\newcommand{\BuiltInTok}[1]{#1}
\newcommand{\CharTok}[1]{\textcolor[rgb]{0.31,0.60,0.02}{#1}}
\newcommand{\CommentTok}[1]{\textcolor[rgb]{0.56,0.35,0.01}{\textit{#1}}}
\newcommand{\CommentVarTok}[1]{\textcolor[rgb]{0.56,0.35,0.01}{\textbf{\textit{#1}}}}
\newcommand{\ConstantTok}[1]{\textcolor[rgb]{0.56,0.35,0.01}{#1}}
\newcommand{\ControlFlowTok}[1]{\textcolor[rgb]{0.13,0.29,0.53}{\textbf{#1}}}
\newcommand{\DataTypeTok}[1]{\textcolor[rgb]{0.13,0.29,0.53}{#1}}
\newcommand{\DecValTok}[1]{\textcolor[rgb]{0.00,0.00,0.81}{#1}}
\newcommand{\DocumentationTok}[1]{\textcolor[rgb]{0.56,0.35,0.01}{\textbf{\textit{#1}}}}
\newcommand{\ErrorTok}[1]{\textcolor[rgb]{0.64,0.00,0.00}{\textbf{#1}}}
\newcommand{\ExtensionTok}[1]{#1}
\newcommand{\FloatTok}[1]{\textcolor[rgb]{0.00,0.00,0.81}{#1}}
\newcommand{\FunctionTok}[1]{\textcolor[rgb]{0.13,0.29,0.53}{\textbf{#1}}}
\newcommand{\ImportTok}[1]{#1}
\newcommand{\InformationTok}[1]{\textcolor[rgb]{0.56,0.35,0.01}{\textbf{\textit{#1}}}}
\newcommand{\KeywordTok}[1]{\textcolor[rgb]{0.13,0.29,0.53}{\textbf{#1}}}
\newcommand{\NormalTok}[1]{#1}
\newcommand{\OperatorTok}[1]{\textcolor[rgb]{0.81,0.36,0.00}{\textbf{#1}}}
\newcommand{\OtherTok}[1]{\textcolor[rgb]{0.56,0.35,0.01}{#1}}
\newcommand{\PreprocessorTok}[1]{\textcolor[rgb]{0.56,0.35,0.01}{\textit{#1}}}
\newcommand{\RegionMarkerTok}[1]{#1}
\newcommand{\SpecialCharTok}[1]{\textcolor[rgb]{0.81,0.36,0.00}{\textbf{#1}}}
\newcommand{\SpecialStringTok}[1]{\textcolor[rgb]{0.31,0.60,0.02}{#1}}
\newcommand{\StringTok}[1]{\textcolor[rgb]{0.31,0.60,0.02}{#1}}
\newcommand{\VariableTok}[1]{\textcolor[rgb]{0.00,0.00,0.00}{#1}}
\newcommand{\VerbatimStringTok}[1]{\textcolor[rgb]{0.31,0.60,0.02}{#1}}
\newcommand{\WarningTok}[1]{\textcolor[rgb]{0.56,0.35,0.01}{\textbf{\textit{#1}}}}


% ---
% Dados dos documento
% ---

\titulo{Trabalho de Análise de Dados e Comunicação}
% \autor{Nome Aluno 1 \\ Nome Aluno 2}
\autor{João Victor Pietchaki Gonçalves \\ Nome do Aluno 2}
\data{2024}
\instituicao{Universidade Federal do Paraná
             \par
             Setor de Ciências Exatas
             \par
             Departamento de Estatística
%             \par
%             Curso de Estatística
            }
\local{Curitiba}
\orientador[Orientador(a):]{Nome do orientador(a)}
\preambulo{Projeto de Pesquisa apresentado à disciplina Laboratório A do Curso de Graduação em Estatística da
           Universidade Federal do Paraná, como requisito para elaboração do Trabalho de Conclusão de Curso.}

% ---
% Configurações básicas
% ---

% informações do PDF

\makeatletter
\hypersetup{
     	%pagebackref=true,
		pdftitle={\@title},
		pdfauthor={\@author},
    	pdfsubject={\imprimirpreambulo},
		colorlinks=true,       		% false: boxed links; true: colored links
    	linkcolor=black,          	% color of internal links
    	citecolor=black,            % color of links to bibliography
    	filecolor=magenta,      	% color of file links
		urlcolor=black,
		bookmarksdepth=4
}
\makeatother

\setlength\afterchapskip{\lineskip}

% ----
% Início do documento
% ----

\begin{document}
\frenchspacing     % Retira espaço extra obsoleto entre as frases.

% ----------------------------------------------------------
% ELEMENTOS PRÉ-TEXTUAIS
% ----------------------------------------------------------

% ---
% Capa
% ---
\imprimircapa
% ---

% ---
% Folha de rosto
% ---
\imprimirfolhaderosto
% ---

% ---
% inserir o sumario
% ---
\tableofcontents*
\cleardoublepage
% ---

\makepagestyle{abntheadings}
\makeevenhead{abntheadings}{\ABNTEXfontereduzida\thepage}{}{}
\makeoddhead{abntheadings}{}{}{\ABNTEXfontereduzida\thepage}
\makeheadrule{abntheadings}{\textwidth}{0in}

% ----------------------------------------------------------
% ELEMENTOS TEXTUAIS
% ----------------------------------------------------------
\textual

\chapter{Introdução}\label{introduuxe7uxe3o}

\bigskip

Digite a introdução do projeto.

Para citar referências, basta usar a sintaxe \texttt{\textbackslash{}cite\{key\}} para citação
indireta, ou \texttt{\textbackslash{}citeonline\{key\}} para citação direta, onde \texttt{key} é a
chave para a referência.

Por exemplo, podemos citar no texto, segundo \citeonline{kaplan}. No
entanto, podemos deixar para fazer a citação no final da frase
\cite{Casella-Berger2011}.

Algumas outras referências são \citeonline{Wilks1962} e
\citeonline{Mood1974}.

O principal resultado do projeto é obter um gráfico tão impressionante
quanto aquele que está representado na Figura \ref{fig:disp}. Note que,
para figuras, o label do chunk vira a referência. Aqui esse label é
\texttt{disp}, portanto, a referência para a figura fica \texttt{fig:disp}, e usamos
\texttt{\textbackslash{}@ref(fig:disp)}. Outra coisa importante é que, para isso funcionar, é
necessário obrigatoriamente especificar uma legenda nas próprias opções
do chunk, com a opção \texttt{fig.cap}.

\begin{figure}[!h]

{\centering \includegraphics[width=0.8\linewidth]{C:/Users/joaov/Downloads/CE302-Data-Analysis-And-Communication/03_results/031_report/Data-Analysis-Report_files/figure-latex/disp-1} 

}

\caption{Uma legenda para esse gráfico.}\label{fig:disp}
\end{figure}

Uma figura externa também pode ser incluída. Nesse caso, a melhor opção
é usar a função \texttt{include\_graphics()} do \textbf{knitr}, e controlar a
aparência com as opções do chunk. Veja um exemplo na Figura
\ref{fig:leg}.

\begin{figure}

{\centering \includegraphics[width=0.1\linewidth]{logo/leg} 

}

\caption{O logo do LEG.}\label{fig:leg}
\end{figure}

Também é possível incluir códigos, se for necessário. Veja no próximo
parágrafo como isso funciona.

Uma descrição da base de dados \texttt{iris} pode ser obtida com a função
\texttt{summary()}, que faz um resumo estatístico todas as variáveis presentes
em um objeto da classe \texttt{data.frame}. Veja o resultado da chamada dessa
função abaixo.

\begin{Shaded}
\begin{Highlighting}[]
\FunctionTok{summary}\NormalTok{(iris)}
\end{Highlighting}
\end{Shaded}

\begin{verbatim}
##   Sepal.Length    Sepal.Width     Petal.Length    Petal.Width          Species  
##  Min.   :4.300   Min.   :2.000   Min.   :1.000   Min.   :0.100   setosa    :50  
##  1st Qu.:5.100   1st Qu.:2.800   1st Qu.:1.600   1st Qu.:0.300   versicolor:50  
##  Median :5.800   Median :3.000   Median :4.350   Median :1.300   virginica :50  
##  Mean   :5.843   Mean   :3.057   Mean   :3.758   Mean   :1.199                  
##  3rd Qu.:6.400   3rd Qu.:3.300   3rd Qu.:5.100   3rd Qu.:1.800                  
##  Max.   :7.900   Max.   :4.400   Max.   :6.900   Max.   :2.500
\end{verbatim}

\chapter{Objetivos}\label{objetivos}

\bigskip

\section{Objetivos Gerais}\label{objetivos-gerais}

Analisar os dados do \ldots{}

\section{Objetivos Específicos}\label{objetivos-especuxedficos}

\begin{enumerate}
\def\labelenumi{\alph{enumi}.}
\tightlist
\item
  Identificar \ldots{}
\item
  Estudar \ldots{}
\item
  Discutir \ldots{}
\end{enumerate}

\chapter{Material e Métodos}\label{material-e-muxe9todos}

\bigskip

\section{Material}\label{material}

Descrever os dados e softwares a serem utilizados para a análise dos
dados.

Os dados podem ser apresentados em uma tabela, que pode ser
referenciada. Por exemplo, veja a Tabela \ref{tab:dados}.

\begin{table}[ht]
\centering
\caption{Uma legenda para essa tabela com \textbf{xtable}.} 
\label{tab:dados}
\begin{tabular}{rrrrl}
  \hline
Sepal.Length & Sepal.Width & Petal.Length & Petal.Width & Species \\ 
  \hline
5.10 & 3.50 & 1.40 & 0.20 & setosa \\ 
  4.90 & 3.00 & 1.40 & 0.20 & setosa \\ 
  4.70 & 3.20 & 1.30 & 0.20 & setosa \\ 
  4.60 & 3.10 & 1.50 & 0.20 & setosa \\ 
  5.00 & 3.60 & 1.40 & 0.20 & setosa \\ 
  5.40 & 3.90 & 1.70 & 0.40 & setosa \\ 
   \hline
\end{tabular}
\end{table}

Note que a tabela acima foi gerada usando o pacote \textbf{xtable}
\cite{xtable}, que funciona bem para \LaTeX{}, mas pode não ser portável
caso queira utilizar o mesmo texto em uma página HTML, por exemplo. Por
isso, a mesma tabela pode também ser gerada pela função
\texttt{knitr::kable()}. Note que agora, o \emph{label} de referência é o próprio
nome do chunk, com o prefixo \texttt{tab:}, veja Tabela \ref{tab:dados2}. Para
mais opções de tabelas, consulte o pacote \textbf{kableExtra} \cite{kableExtra}.

\begin{table}

\caption{\label{tab:dados2}Uma legenda para essa tabela com \textbf{kable}.}
\centering
\begin{tabular}[t]{r|r|r|r|l}
\hline
Sepal.Length & Sepal.Width & Petal.Length & Petal.Width & Species\\
\hline
5.1 & 3.5 & 1.4 & 0.2 & setosa\\
\hline
4.9 & 3.0 & 1.4 & 0.2 & setosa\\
\hline
4.7 & 3.2 & 1.3 & 0.2 & setosa\\
\hline
4.6 & 3.1 & 1.5 & 0.2 & setosa\\
\hline
5.0 & 3.6 & 1.4 & 0.2 & setosa\\
\hline
5.4 & 3.9 & 1.7 & 0.4 & setosa\\
\hline
\end{tabular}
\end{table}

Evite dizer que uma tabela está ``abaixo'' ou ``acima''. Aqui, por exemplo,
a tabela está abaixo do parágrafo, mas no documento compilado ela
aparece depois de outro parágrafo.

Esse é mais um texto só para empurrar a próxima sessão para baixo.
Aproveito para citar mais um artigo de \citeonline{Bonat2018}, e outro
no final do parágrafo \cite{OHara2009}.

\section{Métodos}\label{muxe9todos}

Descrever os métodos que pretende utilizar. Tente ser objetivo, focando
no método específico que irá utilizar. Uma descrição geral do método
deve ser incluida na introdução, como revisão de literatura.

Equações matemáticas funcionam normalmente com a sintaxe do \LaTeX{},
como por exemplo

\begin{equation*}
P(X = x) = \frac{e^{-\lambda} \lambda^x}{x!},
  \quad x = 0, 1, 2, \ldots .
\end{equation*}

As equações também podem ser referenciadas no texto, bastando adicionar
um label no formato \texttt{(\textbackslash{}\#eq:binom)}, como por exemplo

\begin{equation}
f\left(k\right) = \binom{n}{k} p^k\left(1-p\right)^{n-k}.
\label{eq:binom}
\end{equation}

Para referenciar a equação \eqref{eq:binom}, use \texttt{\textbackslash{}@ref(eq:binom)}.

Existem várias opções de ambientes, inclusive para definições, teoremas
e provas. Veja a página do Bookdown\footnote{\url{https://bookdown.org/yihui/bookdown/markdown-extensions-by-bookdown.html\#equations}}.

\begin{landscape}
\chapter{Cronograma de atividades}
\bigskip


% \begin{landscape}    % página na orientação paisagem

\begin{table}[h]
\begin{tabular}{lllllll}
\hline
 & \textbf{ATIVIDADES} & \textbf{02/2016} & \textbf{03/2016} & \textbf{04/2016} & \textbf{05/2016} & \textbf{06/2016} \\ \hline
\textbf{1} & \textbf{Projeto de Pesquisa} &  &  &  &  &  \\
 & Entrega da versão final do Projeto de Pesquisa ao orientador & \cellcolor[HTML]{C0C0C0}{\color[HTML]{C0C0C0} } &  &  &  &  \\
\textbf{2} & \textbf{Elaboração do Trabalho de Conclusão de Curso} &  &  &  &  &  \\
 & Revisão de literatura sobre o tema & \cellcolor[HTML]{C0C0C0} & \cellcolor[HTML]{C0C0C0} &  &  &  \\
 & Análise dos dados e discussão dos resultados obtidos &  & \cellcolor[HTML]{C0C0C0} & \cellcolor[HTML]{C0C0C0} & \cellcolor[HTML]{C0C0C0} &  \\
 & Redação do trabalho de conclusão de curso &  &  & \cellcolor[HTML]{C0C0C0} & \cellcolor[HTML]{C0C0C0} &  \\
 & Leitura do trabalho pelo orientador e correções &  &  &  & \cellcolor[HTML]{C0C0C0} & \cellcolor[HTML]{C0C0C0} \\
 & Entrega do trabalho redigido aos membros da banca &  &  &  &  & \cellcolor[HTML]{C0C0C0} \\
\textbf{3} & \textbf{Defesa do Trabalho de Conclusão de Curso} &  &  &  &  &  \\
 & Preparação e apresentação do trabalho de conclusão de curso &  &  &  &  & \cellcolor[HTML]{C0C0C0} \\
\textbf{4} & \textbf{Elaboração da Versão Final do Trabalho de Conclusão de Curso} &  &  &  &  &  \\
 & Elaboração da versão final do TCC &  &  &  &  & \cellcolor[HTML]{C0C0C0} \\
 & Entrega da versão final do trabalho ao orientador &  &  &  &  & \cellcolor[HTML]{C0C0C0} \\ \hline
\end{tabular}
\end{table}
\end{landscape}

\setlength{\afterchapskip}{\baselineskip}
\bibliography{bib/ref.bib}

\postextual

% ----------------------------------------------------------
% Referências bibliográficas
% ----------------------------------------------------------

% \setlength{\afterchapskip}{\baselineskip}
% \bibliography{}

% ----------------------------------------------------------
% ELEMENTOS PÓS-TEXTUAIS
% ----------------------------------------------------------
% \postextual
% ----------------------------------------------------------


\end{document}
